\documentclass[aps,prd,twocolumn,floatfix,preprintnumbers,altaffilletter,superscriptaddress]{revtex4-1}
\usepackage{graphicx,url,amssymb,amsmath,rotating,color,units,wasysym,epsfig,multirow,epstopdf,subcaption}
\usepackage[colorlinks,urlcolor=blue,citecolor=blue,linkcolor=blue]{hyperref}
\usepackage{soul,xcolor}

%%% Custom Commands %%%
\newcommand{\TODO}[1]{\textbf{[}{\color{red}TODO: }{\color{blue}#1}\textbf{]}}
\newcommand{\citeme}{\TODO{CITEME}}
\def\code#1{\texttt{#1}}

\begin{document}

\title{Tests of General Relativity with Astrophysical Gravitational Wave Transients}
\author{Noah E. Wolfe}
\date{6 May 2021}

\begin{abstract}
    \TODO{this}
\end{abstract}

\maketitle

\section{Introduction} \label{sec:intro}

% broad introduction of like... what the issue is, why do we care

% Then, begin to focus in on the possibility of using astrophysical GW sources, and why it's relevant (tests have already passed in weak field, non-dynamical regime... and provide citations... now time to look in the strong-field, highly-dynamical regime)

% introduce BBH acroynm up here

% Then...
In the remainder of Section \ref{sec:intro}, we describe the three phases of a BBH merger (Section \ref{sec:intro_anatomy-bbh}), \TODO{number} of the leading alternative theories to general relativity that have been tested thus far (Section \ref{sec:intro_alt-theories}), and a parametric framework for quantifying deviations from general relativity (Section \ref{sec:intro_ppE}). \TODO{maybe add something about how sections II, III, IV each describe some characteristic examples of the tests we can do in each phase...} In Section \ref{sec:inspiral_pn_expansion}, we describe the application of a parametric post-Newtonian expansion to gravitational wave emission in the inspiral phase, and modifications at various orders provided by alternative theories of gravity. In Section \ref{sec:nr_merger}, we describe the challenges involved in accurately modeling the merger phase, and highlight some of the numerical techniques used to solve fully non-linear theories of gravity in order to generate gravitational wave predictions from the merger phase. In Section \ref{sec:ringdown_qnms}, we describe the "quasi-normal modes" of the merged black hole as it relaxes asymptotically to a Kerr state and modifications to these modes under different theories. In Section \ref{sec:current_obsv}, we discuss current comparisons with observations and briefly mention some of the statistical methods used to perform these tests. Finally, in Section \ref{sec:conclusion}, we provide a summary of the work described and a discussion on how future improvements, both in theoretical understanding and observational capacity, will allow us to provide rigorous constraints on deviations from general relativity. \cite{2020arXiv201014529T}

\subsection{Anatomy of a Binary Black Hole Merger} \label{sec:intro_anatomy-bbh}

\subsection{Alternatives to General Relativity} \label{sec:intro_alt-theories}
A number of alternative theories to general relativity may be better described as modifications or extensions of Einstein's theory. 

% dCS
% Einstein dilaton Gauss-Bonnett
% a third one?
% ref something more recent thant he appendix of Yunes, Pratorius... but from those same guys

\subsection{Parameterized Post-Einsteinian Framework} \label{sec:intro_ppE}

\section{Inspiral Phase Post-Newtonian Expansion} \label{sec:inspiral_pn_expansion}

\section{Merger Phase Numerical Relativity} \label{sec:nr_merger}

\section{Ringdown Phase Quasi-Normal Modes} \label{sec:ringdown_qnms}

\section{Current Observations} \label{sec:current_obsv}

\section{Conclusions} \label{sec:conclusion}

% talk about NS mergers in here, maybe?

\bibliography{refs}{}
\bibliographystyle{plain}

\end{document}
